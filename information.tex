\begin{tabular}{|p{2cm}|p{3.5cm}|p{2cm}|}
    \hline
    \textbf{Maximum Value ($N$)} & \textbf{Number with Most Divisors ($n$)} & \textbf{Number of Divisors ($\tau(n)$)} \\
    \hline
    $10^3$ & 83,160 & 128 \\
    \hline
    $10^6$ & 720,720 & 240 \\
    \hline
    $10^7$ & $9,609,600$ & 640 \\
    \hline
    $10^8$ & $98,280,000$ & 672 \\
    \hline
    $10^9$ & $735,134,400$ & 1,344 \\
    \hline
    $10^{10}$ & $7,242,460,800$ & 2,688 \\
    \hline
    $10^{11}$ & $73,346,256,000$ & 5,376 \\
    \hline
    $10^{12}$ & $936,966,912,400$ & 10,752 \\
    \hline
\end{tabular}

\bigskip

\noindent
$\gcd(a,b)=g$ implies there exist integers $x,y$ such that $ax+by=g$.

\begin{itemize}
    \item All integers of the form $ax+by$ are exactly the multiples of $g$.
    \item Adding or subtracting multiples doesn’t change the gcd: $a\equiv b\pmod g \iff g\mid(a-b)$.
    \item If $\gcd(a,b)=1$ then any integer can be formed; if $\gcd(a,b)=g$ then any multiple of $g$ can be formed.
    \item $\gcd(a,b)=\gcd(a-b,b)=\gcd(a,b-a)$.
    \item If $\gcd(a,b)=g$ then $\gcd\bigl(\tfrac{a}{g},\tfrac{b}{g}\bigr)=1$.
    \item $\gcd(ka,kb)=k\gcd(a,b)$.
    \item If $\gcd(a,m)=1$, Bézout gives $ax+my=1$, hence $ax\equiv1\pmod m$, so $x$ is the modular inverse of $a\bmod m$ (important when modulus is needed and $m$ is not prime).
\end{itemize}